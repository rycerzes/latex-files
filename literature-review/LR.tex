\documentclass{article}

% Packages for better formatting and bibliography management
\usepackage[utf8]{inputenc}
\usepackage[english]{babel}
\usepackage{natbib} % For bibliography and citation styles
\usepackage{url}    % For handling URLs
\usepackage{geometry} % For setting page margins
\usepackage{graphicx} % For including images
\usepackage{setspace} % For controlling line spacing
\usepackage{titling}  % For customizing title formatting
\usepackage{enumitem} % For customizing lists
\usepackage{lipsum}   % For generating placeholder text; you can remove this in your final document

% Customizing title formatting
\pretitle{\begin{center}\LARGE\bfseries}
\posttitle{\end{center}\vspace{-1.5em}}

% Set page margins (adjust as needed)
\geometry{margin=1in}

% Title and author information
\title{Usage of the Metaverse in Education}
\author{Swapnil Dutta}
\date{\today}

\begin{document}

\maketitle

\begin{abstract}
The metaverse is a rapidly developing virtual world that has the potential to revolutionize education. This literature review examines the definition, roles, and potential research issues of the metaverse in education. The metaverse is defined as a persistent, online, three-dimensional world that combines elements of social media, online gaming, augmented reality, and virtual reality. The metaverse has the potential to transform education by providing students with immersive and interactive learning experiences that are not possible in traditional classrooms.
\end{abstract}

\section{Introduction}
The metaverse is a rapidly developing virtual world that has the potential to revolutionize education. Artificial intelligence (AI) is another emerging technology with the potential to transform the way we learn and teach.

This literature review examines the definition, roles, and potential research issues of the metaverse in education, with a focus on the role of AI.
\section{Research Questions}
State your research questions or objectives.

\section{Search Methodology}
Describe your methodology for searching and selecting literature.

\section{Key Concepts}
Define key concepts or terms relevant to your literature review.

\section{Main Themes}
\subsection{Theme 1}
Summarize key research papers related to Theme 1.

\subsection{Theme 2}
Summarize key research papers related to Theme 2.

\subsection{Theme 3}
Summarize key research papers related to Theme 3.

\section{Synthesis of Literature}
Synthesize and discuss the findings, trends, and gaps in the literature.

\section{Methodological Approaches}
Discuss the research methods and methodologies used in the reviewed papers.

\section{Discussion}
Discuss the implications of the literature for your research or field.

\section{Conclusion}
Summarize the main findings and contributions of the literature review.

% Example citation
\citep{book_example}

% Bibliography (use BibTeX or manually add your references)
\bibliographystyle{apalike} % Choose your preferred citation style
\bibliography{your_references.bib} % Replace 'your_references' with your BibTeX file

\end{document}
